\documentclass[]{article}
\usepackage[a4paper, total={6in, 8in}]{geometry}

\begin{document}
	\section{Types of Data}
	\textbf{Numeric/Quantitative} -- consists of numbers representing counts or measurements
	\begin{itemize}
		\item \textbf{\textit{Continuous} --} results from infinitely many possible quantitative values, where the collection of values is not countable.
		\begin{itemize}
			\item \textbf{Examples:} Age, Height, Weight
			\item Volume, Temperature, Distance, Time, Speed, Acceleration
		\end{itemize}
		\item \textbf{\textit{Discrete} --} results when the data values are quantitative and the number of values are countable or finite.
		\begin{itemize}
			\item \textbf{Examples:} Number of students in a classroom, number of defective parts
			\item Number of quarters in a piggy bank, number of calls coming into a call center 
		\end{itemize}
	\end{itemize}
	\textbf{Categorical/Qualitative} -- consists of names or labels that are not numbers representing counts or measurements
	\begin{itemize}
		\item \textbf{\textit{Nominal} --} characterized by data that consists of names, labels, or categories only.
		\item Data cannot be arranged in an ordering scheme
		\begin{itemize}
			\item \textbf{Examples:} Color of eyes, religious affiliation, blood type, marital status, cuisine type, car model, operating system
		\end{itemize}
		\item \textbf{\textit{Ordinal} --} can be arranged in some order
		\item Differences between data values have meaning or can be calculated
		\begin{itemize}
			\item \textbf{Examples:} Grades of A,B,C,etc., Critics list of the top 50 movies of all times, satisfaction ratings, pain scale
		\end{itemize}
	\end{itemize}
	
	\section{Design of Experiments (DOE)}
	\subsection{What is a designed experiment?}
	\begin{itemize}
		\item Powerful data collection and analysis tool.
		\item Investigator observes how a response variable (yield, strength, cost) behaves when the researcher \textbf{manipulates} one or more factors/treatments (temperature, speed, vendor)
		\item[$\cdot$] \textbf{Goal:} To determine the effect of explanatory variables/factors on the response
		\item[$\cdot$] \textbf{Questions you are answering with DOE:}
		\begin{itemize}
			\item[$\cdot$] What happens when...?
			\item[$\cdot$] What is the effect of...?
		\end{itemize}
	\end{itemize}
	
	\subsubsection*{Why designed experiments?}
	\begin{itemize}
		\item Causality -- can confidently attribute observed changes to specific factors
		\item Minimize Bias -- ensures any difference in the groups are due to chance
		\item Cost and Time Efficiency -- using a well designed experiment researcher can save time and money to ensure good data
		\item Increase reliability and validity -- reduces errors, controls for confounding variables, and allows for replication
		\item Can generalize to the greater population
	\end{itemize}
	
	\subsection{Observational Studies}
	\subsubsection{What is an Observational Study?}
	\begin{itemize}
		\item Investigators \textbf{observe} characteristics of the sample without assigning treatments or intervention to draw conclusions about corresponding populations or differences between two or more groups.
		\item Retrospective -- Use either all or a small sample of historical data from some time period.
		\item \textbf{Goal:} To describe a group or situation, identify associations between variables, and understand natural behaviors or outcomes without researchers intervening or manipulating anything.
	\end{itemize}
	\subsubsection{Why an Observational Study?}
	\begin{itemize}
		\item \textbf{Ethical Reasons:} It's unethical to assign certain risks or treatments to people in a controlled experiment.
		\item \textbf{Practicality:} Ideal for studying the causes of conditions or diseases, especially rare conditions.
		\item \textbf{Real-World Settings:} Allows for observations in a natural environment, showing how people make choices and react in their real lives.
	\end{itemize}
	\subsubsection{Cons of an Observational Study}
	\begin{itemize}
		\item Lack of randomization of experimental units (subjects)
		\subitem Ethical issues (e.g. Seatbelts)
		\subitem Traits not randomly assigned
		\item \textbf{Causality:} Cannot make cause and effect conclusions
		\item \textbf{Confounding Factors:} Cannot rule out the possibility that observed effect is due to some other variable.
		\item \underline{Not able to generalize} to the greater population.
	\end{itemize}
	\subsection{Principles of Good Experiments}
	\subsubsection{Randomization}
	\begin{itemize}
		\item Experimental units are assigned to treatments at random using some sort of chance process
		\item Equalizes the treatment groups so any differences in the response can be attributed to the explanatory variable (treatments)
		\item Reduces confounding and bias
	\end{itemize}
	In \underline{completely randomized design}, the treatments are assigned to all the experimental units completely by chance.
	\subsubsection{Block}
	A \textbf{block} is a technique used to deal with nuisance variables.
	\begin{itemize}
		\item A \textbf{nuisance} factor is a factor that has some effect on the response, but is of no interest to the experimenter.
		\item However, the variability it transmits to the response needs to be minimized or explained.
		\item \textbf{Examples of Nuisance Factors:}
		\subitem \textit{batches} of raw material if you are in a production situation
		\subitem different \textit{operators}, nurses or subjects in studies
		\subitem the \textit{pieces} of test equipment, when studying a process
		\subitem time (shifts, days, etc.) where the time of day or the shift can be a factor that influences the response.
		\item Random assignment of experimental units to treatments is carried out separately within each block
		\subitem Think of doing the complete replicate of the basic experiment is conducted in each block
		\item Form blocks based on the most important unavoidable sources of variability (lurking variables) among the experimental units.
		\item Randomization will average out the effects of the remaining lurking variables and allow an unbiased comparison of the treatments.
	\end{itemize}
	\noindent\textbf{Block Design}\\
	\textbf{Control Group} A group of experimental units are treated the same but do not receive the treatment.
	\begin{itemize}
		\item Baseline
		\item They may receive no treatment or a placebo.
	\end{itemize}
	\textbf{Placebo:} A treatment which lacks the active ingredient of the treatment being tested, but identical in appearance to the treatment. Placebos are a way of achieving blinding.\\
	\textbf{Placebo effect} -- participants who are given a placebo treatment experience a change even though they are not receiving any active treatment.\\
	\textbf{Blinding:} Participants do not know which treatment they are getting.
	\begin{itemize}
		\item Single or double blinding
		\item Helps reduce bias
		\item Example of blinding without a placebo:
		\subitem Auditions for symphony orchestras take place behind a curtain so that the judges cannot see the performer
		\subitem Blinding the judges to the gender of the performers has been shown to increase the hiring of women
		\subitem Blind tests can also be used to compare the quality of musical instruments
	\end{itemize}
	\subsection{Types of Variables in a DOE}
	\subsubsection{Explanatory Variables}
	\textbf{Explanatory Variables/factors} is the one that may cause or explain the differences in a \textbf{response variable}.
	\subsubsection{Confounding Variable}
	\begin{itemize}
		\item A variable that is in the study and is related to the other study variables, thus having an effect on the relationship between these variables.
		\item One whose effects on the response variable cannot be distinguished from one or more of the explanatory variables in the study
		\item From poor design or cannot avoid
		\item Effects can be neutralized by using DOE techniques.
	\end{itemize}
	\subsubsection{Lurking Variable}
	\begin{itemize}
		\item A variable that is neither the explanatory variable nor the response variable but has a relationship (e.g. may be correlated) with the response and the explanatory variable. 
		\item It is not considered in the study but could influence the relationship between the variables in the study. 
		\item Apparent association between two variables is really just a common response to a third unseen variable. 
		\item A lurking variable, if included in the study, could have a confounding effect and then be classified as a confounding variable. 
		\item Common in observational studies 
		\item Risk if do not randomize properly in a designed experiment. 
	\end{itemize}
	
	
\end{document}