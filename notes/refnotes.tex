\documentclass[]{article}
\usepackage[a4paper, total={6in, 8in}]{geometry}
\usepackage{enumitem}

% A document which is more condensed for definitions and reference

\begin{document}
	\section{Types of Data}
	\begin{description}[font=\sffamily\bfseries, leftmargin=1cm, style=nextline]
		\item[Numeric/Quantitative]
		Consists of numbers representing counts or measurements. \textit{\textbf{Continuous}} values have (potentially) infinite precision/\# of values, \textit{\textbf{Discrete}} values are countable/finite/indivisible.
		\item[Categorical/Qualitative]
		Consists of names or labels that are not numbers representing counts or measurements. \textit{\textbf{Nominal}} values are names/labels/categories \textit{only}. \textit{\textbf{Ordinal}} values can be arranged in some inherent order.
	\end{description}
	
	\section{Design of Experiments (DOE)}

\end{document}