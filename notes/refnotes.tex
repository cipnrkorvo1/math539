\documentclass[]{article}
\usepackage[a4paper, total={6in, 8in}]{geometry}
\usepackage{enumitem}
\usepackage{amsmath}

% A document which is more condensed for definitions and reference

\newenvironment{definitions}{
	\begin{description}[font=\sffamily\bfseries, leftmargin=1cm, style=nextline]
	}{
	\end{description}
}

\begin{document}
	\section{Types of Data}
	\begin{definitions}
		\item[Numeric/Quantitative]
		Consists of numbers representing counts or measurements. \textit{\textbf{Continuous}} values have (potentially) infinite precision/\# of values, \textit{\textbf{Discrete}} values are countable/finite/indivisible.
		\item[Categorical/Qualitative]
		Consists of names or labels that are not numbers representing counts or measurements. \textit{\textbf{Nominal}} values are names/labels/categories \textit{only}. \textit{\textbf{Ordinal}} values can be arranged in some inherent order.
	\end{definitions}
	
	\section{Design of Experiments (DOE)}
	\subsection{Types of Experiments}
	\begin{definitions}
		\item[Designed Experiment]
		Investigator observes how a response variable behaves when a researcher manipulates one or more factors/treatments. Able to generalize to the greater population.
		\item[Observational Study]
		Investigators observe characteristics of the sample without assigning treatments or intervention to draw conclusions about corresponding populations. NOT able to generalize to the greater population.
		\item[Causality]
		Cannot make cause and effect conclusions.
		\item[Confounding Factors]
		Cannot rule out the possibility that the observed effect is due to some other variable.
	\end{definitions}
	\subsection{Types of Variables}
	\begin{definitions}
		\item[Response Variable]
		The target variable of an experiment.
		\item[Explanatory Variable]
		May cause of explain the differences in a response variable.
		\item[Nuisance Variable]
		A variable which has an effect on the response, but is of no interest to the experimenter.
		\item[Confounding Variable]
		One whose effects on the response variable cannot be distinguished from explanatory variables in the study. A type of nuisance variable.
		\item[Lurking Variable]
		Neither explanatory nor the response variable, but may be correlated with both; Not considered in the study but may influence the relationship between the explanatory and response variable. A type of nuisance variable.
		\item[Treatment]
		Combination of factors or explanatory variable levels.
	\end{definitions}
	\subsection{Principles of Good Experiments}
	\begin{definitions}
		\item[Blocking]
		A technique used to deal with nuisance factors/variables.
		\item[Control Group]
		A group of experimental units are treated the same but do not receive the treatment.
		\item[Placebo]
		A treatment which lacks the active ingredient/factor being tested, but is indistinguishable to the participants.
		\item[Single/Double Blinding]
		Participants do not know which treatment they are getting. Double blinded $\rightarrow$ researchers do not know either.
	\end{definitions}
	\subsection{Picking a Sample}
	\subsubsection{Types of Bias}
	\begin{definitions}
		\item[Bias]
		The design of a sample is biased if it systematically favors certain outcomes.
		\item[Selection bias]
		Occurs when some groups in the population are left out because of the process of choosing the sample.
		\item[Nonresponse]
		Occurs when an individual chosen for the sample can't be contacted or refuses to participate.
		\item[Response bias]
		A systematic pattern of incorrect responses, comes from improper execution (poor question wording, tool not calibrated correctly).
	\end{definitions}
	\subsubsection{Types of Sampling}
	\begin{definitions}
		\item[Voluntary Response Sample (-)]
		Consists of people who choose themselves by responding to a general appeal. VRS shows bias because people with strong opinions are most likely to respond.
		\item[Sample of Convenience (-)]
		Researchers use subjects that are near or available to participate in their study.
		\item[Simple random sampling ($\cdot$)]
		Picking individuals from a population such that each individual has an equal chance to be selected.
		\item[Stratified sampling (+)]
		Dividing a population into homogeneous subgroups (\textbf{strata}) based on \textit{relevant characteristics}, and then randomly selecting samples from each stratum. Increases precision, more implementation complexity.
		\item[Cluster sampling (+)]
		Entire population is divided into groups called clusters, and then a random selection of these clusters is chosen for the study. Simple and cost-effective, but potential for bias if clusters are not representative.
		\item[Systematic sampling ($\cdot$)]
		Sample members from a larger population are selected according to a random starting point but a fixed, periodic interval (e.g. Select every fourth person).
	\end{definitions}
	\section{Exploratory Data Analysis (EDA)}
	\subsection{Numerical Data}
		\begin{definitions}
			\item[Mean]
			Measure of center, average, balance point. Notated as $\bar{x}$. For a finite population with $N$ measurements, the population mean is the average of all individuals. The sample mean of a sample of size $n$ is an estimate of the population mean.
			\begin{align*}
				\text{(generic parameter)\quad} \mu = \frac{\sum_{i=1}^{N}x_i}{N}
				\qquad & \qquad \bar{x} = \frac{\text{sum of observations}}{n} = \frac{x_1+x_2+\dots+x_n}{n}
			\end{align*}
			Preferable to use for reasonably symmetric distributions that don't have outliers.
			\item[Median]
			The midpoint of a distribution, the number such that half of the observations are smaller and the other half are larger.\\
			To find the median, arrange all observations from smallest to largest. The median $M$ is the center observation in the ordered list, or the average of the two centered observations if $n$ is even.\\
			Preferable to use for describing skewed distributions or distributions with outliers.
			\item[Robust/Resistant]
			Not influenced by extreme observations.
			\item[Quartiles]
			Each quartile $Q+n$ is the value in a rank-ordered list where $n$ quarter of the values are at or below it.\\
			The first quartile $Q_1$ is located in the $\frac{1}{4}\ast (n+1)$th position.\\
			The second quartile $Q_2$ is located in the $\frac{1}{2}\ast (n+1)$th position (this is the median).\\
			The third quartile $Q_3$ is located in the $\frac{3}{4}\ast (n+1)$th position.
			\item[Variance]
			When the population is finite and consists of $N$ values, we may define the \textit{population variance} as
			\[\sigma^2 = \frac{\sum_{i=1}^{N}(x_i - \mu)^2}{N-1}\]
			The population variance is the average squared distance an observation is from the mean.
			\item[Standard Deviation]
			When the population is finite and consists of $N$ values, we may define the \textit{population standard deviation} as
			\[\sigma = \sqrt{\sigma^2}\]
			The population standard deviation is the average distance an observation is from the mean.\\
			A quick rule of thumb is that the lower/upper bound for usual values is:
			\begin{align*}
				&\text{Lower bound } = \text{ Min Usual Value } = \bar{x} - 2\cdot \sigma\\
				&\text{Upper bound } = \text { Max Usual Value } = \bar{x} + 2\cdot \sigma
			\end{align*}
			\item[Interquartile Range (IQR)]
			The spread of half of your data; The range from the first quartile to the third quartile.
			\[ \text{IQR } = Q_3 - Q_1\]
		\end{definitions}
	\section{EDA -- Graphical}
	\begin{definitions}
		\item[Distribution]
		The shape of the data over the range of values. The distributions of a \textit{quantitative} variable tells us what values the variable takes on and how often it takes those values.
		\item[Center]
		Representative value that indicates where the middle of the data is located.
		\item[Variation]
		Measure of the amount that the data values vary.
		\item[Outliers]
		Values that lie very far away from the vast majority of the data.
		\item[Relationships]
		Associations between variables
		\item[Trends]
		Change in characteristics of the data over time.
	\end{definitions}
	\subsection{Graphs for Quantitative/Numeric Values}
	\begin{definitions}
		\item[Frequency Tables or Distributions]
		Shows how the data are partitioned among several categories by listing the categories along with the number (frequency) of data values in each of them.
		\item[Histograms]
		Shows the distribution of a quantitative variable by using bars whose height represents the number of individuals who take on a variable within a particular class.
		\item[Boxplots]
		A graph of the data using the minimum, maximum, and quartiles.
		\item[Stem and Leaf Plots]
		A graph that divides each data point into a stem and a leaf. It shows the shape of the data using the data itself.
		\item[Dot Plots]
		Displays data where each dot represents one (or more) data point(s).
	\end{definitions}
\section{Probability}
\begin{definitions}
	\item[Experiment]
	An activity that can result in different outcomes, even though it is repeated in the same manner every time. (e.g. flipping a coin, tossing a die)
	\item[Sample Space]
	All possible outcomes of an experiment. \textbf{Discrete} -- finite/countable, \textbf{Continuous} -- interval/uncountably infinite
	\item[Event]
	A subset of a sample space (e.g. rolling two fair dice; let $A$ be the event that the sum of the dice is 8). The probability of an event $A$ occurring:
	\[P(A) = \frac{\text{Number of outcomes in an event}}{\text{Number of outcomes in a sample space}}\]
	\item[Empirical Probability]
	An estimate of the value using empirical data from an experiment/series of experiments.
	\[P(A) = \frac{\text{Number of times the event has occurred}}{\text{Number of times experiment repeated}}\]
	\item[Law of Large Numbers]
	A mathematical law that states that the average of the results obtained from a large number of independent random samples converges to the true value, if it exists.
\end{definitions}
	\subsection{Probability Rules}
	\begin{definitions}
		\item[Rule 1]
		For any event $A$, the range of possible probabilities is $0 \leq P(A) \leq 1$.
		\item[Rule 2]
		For the sample space $S$ of all possibilities, $P(S) = 1$.
		\item[Rule 3: Addition Rule]
		If two events $A$ and $B$ are \textit{mutually exclusive} or \textit{disjoint}, then
		\[P(A \cup B) = P(A) + P(B)\]
		If two events are NOT mutually exclusive, then
		\[P(A \cup B) = P(A) + P(B) - P(A \cap B)\]
		\item[Rule 4: Complement Rule]
		For any event $A$,
		\[P(A^c) = 1 - P(A) \text{ or } P(A) = 1 - P(A^c)\]
		\item[Rule 5: Conditional Probability]
		For two events $A$ and $B$, $P(A|B)$ means the probability that event $A$ occurs \textit{given} event $B$ has occurred.
		\[ P(A|B) = \frac{P(A \cap B)}{P(B)}\]
		\item[Rule 6: Multiplication Rule]
		The probability that both events $A$ and $B$ occur:
		\[P(A \cap B) = P(B) * P(A|B)\]
		If $A$ and $B$ are \textbf{independent}--neither event influences or affects the probability that the other event occurs--then
		\[ P(A \cap B) = P(A) * P(B)\]
		This particular rule extends to more than two independent events:
		\[ P(A \cap B \cap C) = P(A) * P(B) * P(C)\]
	\item[Event Independence]
	Two events $A$ and $B$ are \textit{independent} if any of the following (equivalent) statements are true:
	\[P(B|A) = P(B) \qquad P(A|B) = P(A) \qquad P(A\cap B) = P(A)*P(B)\]
\end{definitions}
\section{Discrete Probability Distributions}
\begin{definitions}
	\item[Probability model]
	Describes the possible outcomes of a chance process and the likelihood that those outcomes will occur.
	\item[Random variable]
	A numerical variable that describes the outcomes of a chance process
	\item[Discrete random variable]
	A discrete random variable (RV), $X$, takes a fixed set of possible values with gaps between.
	\item[Probability mass function]
	The pmf is used to describe the probability distribution of a discrete RV, $X$.\\
	Must satisfy two requirements:
	\[
		0 \leq f(x_i) \leq 1 \hspace{2cm} \sum_{i=1}^{k}f(x_i) = 1
	\]
	To find the probability of any event, add the probabilities $p_i$ of the particular values $x_i$ that make up the event.
	\item[Cumulative distribution function]
	A cdf of a discrete random variable $X$ is the probability that the variable takes a value less than or equal to $x$.
	\[ F(x) = P(X \leq x) = {\sum}_{x_i \leq x}f(x_i) \]
	\item[Expected value]
	The number that we would expect any observation to be near. Population/distribution mean is the same as expected value. Can think of as weighted average.
	\[ E(X) = \mu = \sum_{i=1}^{k}x_if(x_i) \]
	\item[Variance]
	A measure of how much the probability mass is spread around this center.
	\[ V(X) = \sigma^2 = E(X - \mu)^2 = \sum_{i=1}^{k}(x_i-\mu)^2f(x_i) = \left[\sum_{i=1}^{k}x_i^2f(x_i)\right]-\mu^2 \]
\end{definitions}
\subsection{Binomial Distribution}
\begin{definitions}
	\item[Bernoulli trial]
	A single run of an experiment or process which results in only one of two options, usually labeled as "success" or "failure"
	\item[Binomial experiment]
	Consists of $n$ identical and independent Bernoulli trials.
	\item[Binomial random variable]
	Results from a binomial experiment and is defined as a count of the number of "successes" out of $n$ trials.
	\item[Binomial probability distribution]
	A p.d. which results from a procedure or experiment that meets the following requirements:
	\begin{enumerate}[itemsep=0pt]
		\item The procedure has a fixed number of trials.
		\item The trials are independent.
		\item Each trial must have all outcomes classified into two categories (success or failure).
		\item The probability of success remains the same in all trials.
	\end{enumerate}
	\item[Binomial distribution parameters]
	Minimum necessary information to calculate probabilities:
	\begin{itemize}[itemsep=0pt]
		\item $p$ = probability that a randomly selected Bernoulli trial results in a "success"
		\item $n$ = number of independent Bernoulli trials
	\end{itemize}
	\item[Binomial notation]
	\[ X \sim Bin(n,p)\]
	\item[Binomial pmf]
	\begin{align*}
		P(x = x) &= P(x)\\
		&= \binom{n}{x} p^x(1-p)^{n-x}
	\end{align*}
	Where
	\[ \binom{n}{x} = nCx = \frac{n!}{x!(n-x)!} \]
	\item[Binomial expected value]
	\[E(x) = \mu = np\]
	\item[Binomial variance and standard deviation]
	\[ \sigma = \sqrt{V(X)} = \sqrt{npq} \]
	
\end{definitions}

\end{document}